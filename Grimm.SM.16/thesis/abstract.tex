% $Log: abstract.tex,v $
% Revision 1.1  93/05/14  14:56:25  starflt
% Initial revision
% 
% Revision 1.1  90/05/04  10:41:01  lwvanels
% Initial revision
% 
%
%% The text of your abstract and nothing else (other than comments) goes here.
%% It will be single-spaced and the rest of the text that is supposed to go on
%% the abstract page will be generated by the abstractpage environment.  This
%% file should be \input (not \include 'd) from cover.tex.
The research presented in this thesis was inspired by an interest in determining feedback strategies for high-dimensional pursuit-evasion games. When a problem is high-dimensional or involves a state space that is defined by several variables, various methods used to solve pursuit-evasion games often require unrealistic computation time. This problem, called the curse of dimensionality, can be mitigated under certain circumstances by utilizing tensor-train (TT) decomposition. By using this intuition, a new algorithm for solving high dimensional pursuit-evasion problems called Best-Response Tensor-Train-decomposition-based Value Iteration(BR-TT-VI) was developed. BR-TT-VI builds on concepts from game theory, dynamic programming (DP), and tensor-train decomposition. By using TT decomposition, BR-TT-VI greatly reduces the effects of the curse of dimensionality. This work culminates in the application of BR-TT-VI to two different pursuit-evasion problems. First, a four-dimensional problem capable of being solved by traditional value iteration(VI) is tackled by the BR-TT-VI algorithm. This problem allows a direct comparison between VI and BR-TT-VI to demonstrate the reduced computational time of the new algorithm. Finally, BR-TT-VI is used to solve a six-dimensional problem involving two Dubins vehicles that is impractical to solve with VI.   
