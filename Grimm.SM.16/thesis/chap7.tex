%% This is an example first chapter.  You should put chapter/appendix that you
%% write into a separate file, and add a line \include{yourfilename} to
%% main.tex, where `yourfilename.tex' is the name of the chapter/appendix file.
%% You can process specific files by typing their names in at the 
%% \files=
%% prompt when you run the file main.tex through LaTeX.
\chapter{Conclusion}\label{chp:con}
Best Response TVI and the underlying TVI algorithm have a number of advantages and disadvantages. This chapter will begin by exploring both the advantages and disadvantages of TVI as they relate to solving pursuit-evasion problems. A reflection on the constraints imposed by TVI's disadvantages will also be included in the first section. After detailing the benefits and constraints of TVI, a number of potential improvements for future research will be suggested. Finally, last remarks will be given on the importance of the research presented in this paper.  

\section{Summary of Results}
Tensor-based value iteration's ability to compute a solution in an efficient manner is the method's single greatest benefit. This method reduces the computational time of a four-dimensional problem to approximately a tenth of the time and makes a six-dimensional problem efficiently computable. However, TVI and the Best Response TVI algorithm still have a variety of problems. First, as stated in \Cref{chp:examples}, TVI only provides an approximation which can be held to some accuracy $\epsilon$. While $\epsilon$ can be reduced to some very small value, doing so can reduce the benefits of TVI by resulting in approximate tensors with large tensor ranks. As noted in \Cref{chp:TT}, the larger the ranks become the computational complexity of using tensor-trains increases dramatically. For this reason, using TVI often resulted in a balancing act of accuracy and complexity. Because of this, using TVI often required manually determining an accuracy that would maintain low-rank.

Maintaining low-rank often created restrictions on the modeling of a problem. Creating large capture regions or other discontinuity producing areas often strained TVI in finding a low-rank approximation. This was especially troublesome considering that the approximations often took the largest toll when the pursuer and evader were in close distance of each other. These restrictions required a relatively simple game in which the pursuer tried to minimize the distance between itself and the evader while the evader attempted to maximize this value.

The approximate nature of TVI also created a problem with potentially negative values. Often when the state cost for a particular state was either zero or close to zero, TVI would approximate this to a negative value. These values could potentially spiral out of control making the entirety of the state space converge to $-\infty$ as the number of iterations approached $\infty$. For this reason a constant of 10 was added to the state cost in \Cref{4cost,6cost}.

Despite a variety of shortcomings, Best Response TVI makes numerous problems efficiently solvable as detailed in \Cref{chp:examples}. As noted for the six-dimensional problem, using TVI reduces the time required to find a solution from weeks for traditional value iteration to less than half an hour. For some problems, such as the four-dimensional problem in \Cref{chp:examples}, the results for Best Response TVI and traditional value iteration are indistinguishable with Best Response TVI taking only a tenth of the time.           

\section{Future Directions}
The Best Response TVI algorithm provides a great basis for improvement. As already noted, best response TVI requires manual input of three different accuracy values, one for the tensor, value iteration, and best response. An algorithm capable of adjusting each of these accuracies to ensure the most accurate solution possible without causing either the rank to become too large or for the algorithm to enter a never ending loop would greatly assist in the usability of the algorithm. This modified algorithm could self-adjust to determine the optimal accuracy of the problem without requiring prior research or guess work to determine sufficient accuracies.

Numerous potential methods could be used to increase the accuracy of the solution by combining best response TVI with other methods. One such method could be to use the TVI approximation to create a full state array and continue the algorithm with traditional value iteration. Another method could be to use TVI to approximate the majority of the state space while using traditional value iteration to determine small areas of the state space that either require more precise accuracy or are not naturally low rank such as absorption regions.

Finally there are numerous ways to test best response TVI with a more complex state space or dynamics. This paper has constrained its use of best response TVI to vehicles navigating a two-dimensional space such as with cars. There is potential that best response can be used to navigate three-dimensional space such as with aircraft that can have variable altitude. More complex dynamics could include adding the effects of friction or wind resistance. Applying best response TVI to an environment with obstacles would also require overcoming a variety of complications due to discontinuities in the state space. These are just a few of the many future opportunities to conduct research with best response TVI.    


\section{Final Remarks}
Best Response TVI builds on the work of a variety of methods in order to efficiently solve pursuit-evasion games. The computational efficiency of best response TVI enables platforms with limited computational power to solve complex problems in an efficient manner. This is especially true for autonomous vehicles in pursuit-evasion environments. For example, suppose best response TVI was ran overnight to produce controls for an autonomous spy vehicle to tail a certain target using a vehicle that has Dubins dynamics with a certain maximum speed and turning radius. However, halfway through the mission the target decides to change vehicles to one with a different maximum speed or turning radius. Best response TVI would allow the autonomous spy vehicle to quickly approximate new optimal controls and continue with the mission. Best response TVI holds the potential to solve problems that would take days in hours and problems that would take hours in minutes.       
